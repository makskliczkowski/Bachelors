After the synthesis, the lasting of many redundant impurities inside the solution provides unwanted complications such as lower photo-luminescence, faster oxidation or potential problems with the traps. Therefore, separating our beloved nano-crystals would be crucial for achieving the purity of the device. The  impurities that did not react during the process are a large scale particles that can be  filtered by many different methods. \cite{Shen2017}\cite{purif} To do that, people usually change three main properties of the QDs solution which are: polarity, size and mobility. For any colloidal system, the challenge that we approach is not only connected with the type and the amount of ligands connected to the core crystal but the solvent properties as well so the parameters may change in a real time for any of the part and we need to be really careful not to change any drastically. 

\subsection{Purification method}

For our case of PbS nano-crystals, the method used was a polarity based purification technique. There are two most common processes in which we can clear the untidy system when considering only polarity. It is either precipitation and re-dissolution (PR) method or extraction method. 

The first one is frequently used in case of non-polar solvents and thanks to the anti-solvents introduction the polarity of the solution is increased. Then, the retaliation of the impurities via the supernatant(which is the mixture that is floating after the centrifugation of the suspension) is taken away and we are left with quite pure nano-crystals which can be the redissolved in clean solvents. The repetition of the process is available for us for several times but not infinitely, because we can misleadingly get rid of ligands, which are necessary in chemical non-static processes that occur inside the solvent. It has also been seen for many times, that larger particles are more able to participate in the polarity induced precipitation than the smaller ones which means it's harder to get rid of the later ones. The method isn't always accurate because of the ability of the ligands to dissolve which can be really similar to the nano-crystals. More can be read in publications mentioned in the beginning of the chapter.
\vline

\subsection{Method's properties}
\label{subsection:CHXpurification}

\subsubsection{CHX/Acetonitrirle polar purification.} 


The first thing to do was to add chloroform to the nano-particles dissolved in CHX solution. CHX is the standard cykloalkane used in case of quantum dot nano-crystals. The amount of the liquid was about the former amount of CQDs solution. Then, to begin it's work as a purification mixture, we added polar solvent acetonitrile. The unwanted particles and some of the wrenched ligands are then separated using the centrifuge. Carefully, in the nitrogen created vacuum, we separated the supernatant because of the danger of PbS fast oxidation. Chloroform was then again used as a solvent for now clean CQDs.

