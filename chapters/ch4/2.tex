\subsection{Conclusion and outlook}

We can now see that the big improvement of the device has been made when comparing to the first one. As it was based on the same method and structure, it is most probable that the layer width of the second device is more optimal. Another point is that the time spent on air exposure was probably too large for the first device and it might have suppressed it's properties. Nevertheless, with some optimisation of the technique, the achievement of surpassing the commercial device is undeniable. As it is possibly not the last version of the device and the upgrades are ought to be provided in the future, there are plenty of possibilities to make it better. We have to say, that main problem of the device is a small FF and considerably short lifetime(in terms of device usage). One may think how to improve them. On the other hand, conclusion is that the biggest impact on the PCE of the device has been made by the placement of $V_oc$. We may also note that the stability in the air is due to imperfections of the manufacture, not the parameters of QD sensitizers because the potential stability of this alignment has been proven before\cite{Chuang2014}. Yet, this document also provides us with the information that air stability is essentially better without using the $MoO_x$ layer. We may think about other hole extracting material with comparable optical and electric properties. There is also possibility of adding another layer between as it may provide some improvement and, as the valence band of MoOx depends on concentration of the oxygen we may test different layers of that. Some of the improvements in the PCE may be due to the better alignment of PbS-EDT and PbS-TBAI when compared to the previous one, as it has been proven to be crucial due to band offsets between them that block electrons coming to the anode, when simultaneously finely transporting holes. We can also potentially create better alignment between ZnMgO and PbS-TBAI with changing the Mg concentration(thus changing the optical bandgap. We could try to make other metalic contacts as well, to create better series resistance, as it will improve the FF. Better $I_sc$ will surly be achieved with less defects inside the structure but there was no possibility to see any of the defect states,  because of the absence of such measuring system. From the series resistance we can see that it's effects are smaller than those of the shunt resistance. Therefore we conclude that to achieve a better FF and theoretically better PCE, we have to drastically reduce the shunt resistance(reduce current transport paths, defects etc.). After that, if the sufficiently narrow-luminescence width CQDs is achieved, we can think about multiple exciton generation. Another thing is to think about providing better carrier mobilities and try to extract the carriers before recombining. 