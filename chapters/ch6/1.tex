Providing that the following thesis is only an introduction to the big world of semiconductor photovoltaic devices, there is a huge field of work still to be done. One can say that the information from the chapters included is only a grain of sand in the big dessert, therefore, in this final chapter, some possible areas that can improve the former thoughts will be stated. Promise, it won't be long, as the ideas come and go very fast in the physical world. From the general point of view, there is a lot of things that can be improved in the process of production only. The spin-coating method of creating layers has it's pros and cons. We may think of possible changes in the way of making them. It is rather a functionalisation method which needs to be done after one can make a device which gives some results. Inside this process, there is also some way to manipulate the width of the layers, number of depositions, some new layers maybe. Then, after that, there is probably the urging need to apply new materials. I won't state any publications about that, as it has been stated before, but one might think of creating a hybrid devices, with many different layers of distinct materials as it can deal with some physical problems like recombination processes. Great possibilities approach thereby by designing nanostructure layers capable of multi exciton generation, as it is probably the biggest reason to study those material. To thoughtfully make the application of new materials possible and, of course, rather profitable, in my opinion, the necessity of theoretical work is irrefutable. As for this scale, the nonlinear light processes occur to be very important there is a need to discuss the effects and how to exploit them to our needs. Probably, the improving step would be to analyse firstly the theoretical concepts of transport properties, recombination possibilities and etc., for then to come up with numerical simulations of such materials, as most of the programs available today focus on standard type I solar cells. One can apply the non-equilibrium Green Functions theory for the transport properties (f.e. \cite{green}). For me, it would be really interesting thing to consider, as that opens people's minds for the quicker thinking of new materials. Because of the scale, there is also a great connection of to nano-designing of the structures, to implement 2D confined structures and for sure to use metalic nano-particles, as plasmonics has began to mark it's print on today's physics world(for sure it is useful as many things can be calculated analytically). Nevertheless, I hope that this entry work has pushed up some ideas with it's general form and it will be later developed more...