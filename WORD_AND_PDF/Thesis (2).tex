% Generated by GrindEQ Word-to-LaTeX 
\documentclass{article} %%% use \documentstyle for old LaTeX compilers

\usepackage[english]{babel} %%% 'french', 'german', 'spanish', 'danish', etc.
\usepackage{amssymb}
\usepackage{amsmath}
\usepackage{txfonts}
\usepackage{mathdots}
\usepackage[classicReIm]{kpfonts}
\usepackage[dvips]{graphicx} %%% use 'pdftex' instead of 'dvips' for PDF output

% You can include more LaTeX packages here 


\begin{document}

%\selectlanguage{english} %%% remove comment delimiter ('%') and select language if required

\[20\] 


\noindent 

\noindent 

\noindent 

\noindent 

\noindent \includegraphics*[width=4.41in, height=0.98in, keepaspectratio=false]{image7}

\noindent 

\noindent 

\noindent 

\noindent 

\noindent Faculty of Fundamental Problems of Technology

\noindent Quantum Engineering

\noindent 

\noindent 

\noindent 

\noindent 

\noindent 

\noindent 

\noindent 

\noindent 

\noindent BACHELOR'S THESIS

\noindent 

\noindent \textbf{}

\noindent \textbf{Type III solar cells based on quantum dots}

\noindent \textbf{}

\noindent \textbf{}

\noindent 

\noindent 

\noindent 

\noindent 

\noindent 

\noindent 

\noindent 

\noindent 

\noindent 

\noindent 

\noindent 

\noindent 

\noindent 

\noindent 

\noindent Maksymilian Kliczkowski

\noindent 

 Wroc{\l}aw, data 

\noindent 

\noindent \textbf{}

\noindent \textbf{}

\noindent 

\noindent 

\noindent 

\noindent 

\noindent 

\noindent 

\noindent 

\noindent 

\noindent 

\noindent 

\noindent 

\noindent 

\noindent 
\section{Abstract}

\noindent In this thesis the extensive study of possible and surly optimal materials for Quantum Dot Solar Cells (QDSCs) will be provided to the reader. One will have the opportunity to develop a rather current image of the necessary parts in their architecture which will be supported with an abundant description of the important aspects in engineering a photovoltaic device. With an introduction of its kind we will try to convey a basic but sufficient knowledge of standard terms, light with matter interaction, light spectrum analysis, quantum dots description with their behaviour and a photovoltaic device operation theory with certainly needed quantities that one has to find in order to properly describe a solar cell. In many parts we will try to outline possible enlargement of that information. With the objective of creating a possibly competitive QDSC the description of methods that we used will be included with a related to them characteristics. One will also be able to use this paper as a review of today's development phase and current technological state of the other scientific groups all around the world.  After a successful development of a solar cell, the analysis and comparison will definitely be provided. 

\noindent 

\noindent 

\noindent 

\noindent 

\noindent 

\noindent 

\noindent 

\noindent 

\noindent 

\noindent 

\noindent 

\noindent 

\noindent 

\noindent 
\section{Contents}

\noindent 

Abstract 1Contents 21. Introduction 32. Theory 52.1. Physics and properties of solids and semiconductors 52.2. Classical and semi classical theory of light and light with matter interaction 52.3. Quantum dots and nanostructures 82.4. Introduction to quantum description of light 102.5. Theory of photovoltaic devices 102.6. Solar cell parameters 102.7. Third generation solar cells, QDSCs and review 103. Cell architecture 144. Device depositions and parameters 155. Results analysis 156. Future possibilities 157. Conclusion and outlook 158. References 159. Images 2010. Tables 2011. Attachment 20\textbf{}

\noindent \textbf{\eject }

\noindent \textbf{}


\subsection{ Introduction}

\noindent 

\noindent We, as a society nowadays, are in constant demand for energy. Although, even with the development of calculating, conducting and studying it, we don't really know what the energy is, we are depending on that abstract quantity. One might probably say that to study physics is to endure to study energy in its every possible form. There's a brilliant quote from Bill Bryson that: ``Energy is liberated matter, matter is energy waiting to happen.'' What might be incredible is that from this strictly mathematical quantity we can deduct anything. And what's also important it is as arbitrary as it gets, depending only on one's reference.  From energy we can create few more important quantities such as \textit{power} P, which is simply the energy provided per unit time, so:

\noindent 
\[E=\int P\left(t\right)dt\] 


\noindent The energy will be represented in J (Joules) or eV (\textit{electron volt), }which directly describes energy of elementary charge body ($e\approx 1.602*{10}^{-19}C)$ in 1V (Volt) potential.

\noindent 
\[1eV=1.602*{10}^{-19}J\] 


\noindent Power will be then represented in W (Watts) ($W=\frac{J}{s}\ ).$

\noindent 

\noindent The rise of energy consumption has proven that in the future we will almost certainly require even more. From Global Energy Perspective paper \eqref{GrindEQ__1_} we can learn that:

\begin{enumerate}
\item  Global energy demand will reach plateau at around 2035 despite strong population expansion and economic growth thanks to emphasis on renewable sources, more efficient service industries or more efficient industrial regions

\item  The  energy demand and economic growth became ``decoupled'' for the first time in history

\item  Renewables will provide more than half the electricity after 2035
\end{enumerate}

\noindent The reader is strongly recommended to take a look at the document. 

\noindent 

\noindent With saying that we can produce energy there's a slick trick given with the phrase. Energy cannot be simply created, it is just converted from another source so nothing is ever lost or miraculously made. The problems we need to struggle with are then being able to obtain as much energy from the energy source as it is possible and of course, as humankind society is governed by money, keep the cost lowest. There is plenty of different energy sources that we learned to receive energy from. In (Figure 1) we can see the most vivid ones nowadays. But with the human demand fulfilment come great damages and soon depleting resources of many energy sources such as fossil fuels.

\noindent \includegraphics*[width=6.29in, height=4.44in, keepaspectratio=false]{image8}

\noindent 

\noindent The necessity of searching new possible ways to harness it through renewable sources has become a global issue. Our concern of the environment had never been that serious before. Not only the change of methods for energy production must be enhanced because of this demand but we need to be strictly aware of the word's urging trepidation of Global Warming of which evidence is provided for example here \eqref{GrindEQ__2_} .

\noindent 

\noindent With conventional methods still being the vast majority, the new approaches that might solve most of our problems, are arising to become significant. There are for example:

\noindent 

\begin{enumerate}
\item  hydroelectric farms that use to convert potential energy from water sources and create electrical energy from it with usage of water turbines

\item  wind mills that convert kinetic energy of air flow

\item  tidal energy plants
\end{enumerate}

\noindent 

\noindent Among all of the ideas created in a past few decades, solar energy is believed by many to be the most reliable and promising. It can be directly converted into electricity, heat or chemical energy and our only star seems to be an infinite power resource for us. In the last ten years the world solar PV electricity production has grown impressively, being almost three times bigger in 2016, than in 2010 \eqref{GrindEQ__3_}. In theory, the Sun has the potential to fulfil earth's energy demand if it is not for the technology. Annually, nearly 10${}^{18}$ EJ of energy reaches our planet and of that 10${}^{4}$ EJ is claimed to be harvestable. The possibility of converting the solar energy into electric energy has been studied since discovery of the basic photovoltaic effect and the development of semiconductor technologies.

\noindent \includegraphics*[width=6.00in, height=2.45in, keepaspectratio=false]{image9}

\noindent 


\subsection{ Theory}

\noindent 

\noindent With this chapter the introduction of some theoretical concepts will be provided. Certainly some of chapters below could be omitted without missing the final result and its meaning. Nevertheless, even if the paper is laced with rather practical analysis and effects, a physicist should be sure to understand the aspects of an experiment well, not just to have a better insight into the outcome but also to be sure that during the process no foolish mistakes are made and to be able to find new methods to improve the research. Therefore, few parts will just be a reminder and introduction to notation used or to ensure the understanding and some will be treated as an inquiry of what might be searched to describe it further. 

\noindent 


\paragraph{ Physics and properties of solids and semiconductors}

\noindent 


\subparagraph{ Drude model of free carriers}

\noindent 


\subparagraph{ Crystal structure and Bloch theorem}

\noindent 


\subparagraph{ Energy structure}

\noindent 


\subparagraph{ Bigger world of carries}

\noindent 


\subparagraph{ Non-equilibrium processes }

\noindent 


\subparagraph{ Carrier injection}

\noindent 


\subparagraph{ Recombination}

\noindent 


\subparagraph{ Semiconductor junctions}

\noindent 
\subparagraph{}


\paragraph{ Classical and semi classical theory of light and light with matter interaction}

\noindent 

\noindent As in the photovoltaic physics we are constantly struggling with light itself we should be know what the light actually is and how it interacts with matter in many, rather curious and different ways. Why is that so that matter looks the way it does and what of its properties can we control. In this chapter we will embrace the phenomena just to create an image of what we are dealing with. 

\noindent 


\subparagraph{ Basic properties of electromagnetic field}

\noindent 

\noindent Before we actually begin we need to state some classic information about the physics of electric charges. The electromagnetic field is represented by two generally complex vectors, even though the physical result that we are expecting is ought to be real. Those vectors are \textbf{E }-- \textit{electric field} and \textbf{B }-- \textit{magnetic induction}. The properties of those fields are of course described by the \textit{Maxwell's equations}. For them we shall also introduce two more important quantities $\rho $ -- \textit{the electric charge} density and $\boldsymbol{j}$\textbf{ -- }\textit{electric current density} vector. We can define them in this way:

\noindent 
\[e=\int \rho dV\] 

\[I=\int_S{\boldsymbol{j}\boldsymbol{\bullet }\boldsymbol{dS}}\] 


\noindent Where I is electric current.

\noindent 

\noindent The four Maxwell equations in differential form are:

\noindent 

\noindent $\mathrm{\nabla }\times \boldsymbol{E}\boldsymbol{=\ -}\frac{\partial \boldsymbol{B}}{\partial t}\boldsymbol{\to }$ \textit{Faraday's induction law}

\noindent 

\noindent $\mathrm{\nabla }\times \boldsymbol{B}\boldsymbol{=}{\mu }_0\left(\boldsymbol{j}\boldsymbol{+}{\epsilon }_0\frac{\partial \boldsymbol{E}}{\partial t}\right)\to $ \textit{Ampere's circuital law}

\noindent \textbf{}

\noindent $\mathrm{\nabla }\bullet \boldsymbol{E}\boldsymbol{=}\frac{\rho }{{\epsilon }_0}\boldsymbol{\to }$\textbf{ }\textit{Gauss's law}

\noindent 

\noindent $\mathrm{\nabla }\bullet \boldsymbol{B}\boldsymbol{=\ }\boldsymbol{0}\boldsymbol{\to }$\textbf{ }\textit{Gauss's law for magnetism}

\noindent \textbf{}

\noindent To freely describe the properties of the fields interacting macroscopically with material objects we here we can also introduce standard auxiliary fields with polarization and magnetisation of a macroscopic medium. Those vectors are \textbf{D -- }\textit{the electric displacement} and \textbf{H }-- \textit{the magnetic vector}. From Gauss's law for magnetism there is a straight implication that no magnetics monopoles exist and Gauss's law may be also treated as electric charge density definition. 

\noindent 
\[\boldsymbol{D}\left(\boldsymbol{r},t\right)={\epsilon }_0\boldsymbol{E}\left(\boldsymbol{r},t\right)+\boldsymbol{P}\boldsymbol{(}\boldsymbol{r},t)\] 

\[\boldsymbol{H}\left(\boldsymbol{r},\ t\right)=\frac{1}{{\mu }_0}\boldsymbol{B}\left(\boldsymbol{r}\boldsymbol{,\ }t\right)-M(\boldsymbol{r}\boldsymbol{,\ }t)\] 


\noindent Where \textbf{P }is a \textit{polarization} \textit{vector} and \textbf{M }is \textit{magnetization vector}.

\noindent 

\noindent And with them our former equations change to:

\noindent 

\noindent $\mathrm{\nabla }\times \boldsymbol{E}\boldsymbol{=\ -}\frac{\partial \boldsymbol{B}}{\partial t}\boldsymbol{\to }$ \textit{Faraday's induction law}

\noindent 

\noindent $\mathrm{\nabla }\times \boldsymbol{H}\boldsymbol{=}\left(\boldsymbol{j}\boldsymbol{+}\frac{\partial \boldsymbol{D}}{\partial t}\right)\to $ \textit{Ampere's circuital law}

\noindent \textbf{}

\noindent $\mathrm{\nabla }\bullet \boldsymbol{D}\boldsymbol{=}\rho \boldsymbol{\to }$\textbf{ }\textit{Gauss's law}

\noindent 

\noindent $\mathrm{\nabla }\bullet \boldsymbol{B}\boldsymbol{=\ }\boldsymbol{0}\boldsymbol{\to }$\textbf{ }\textit{Gauss's law for magnetism}

\noindent 

\noindent If we put divergence on the Ampere's law, we can then place Gauss's theorem in the equation because of the exchangeability of partial derivatives and from that we can simply derive so called equation for \textit{charge conservation:}

\noindent \textit{}
\[\frac{\partial \rho }{\partial t}+\mathrm{\nabla }\bullet \boldsymbol{j}\boldsymbol{=}\boldsymbol{0}\] 


\noindent The field is said to be static if all quantities are independent of time and, of course, no currents are present. This is the special case but we cannot be so lucky every time. Optical fields are usually sources of very rapid time variety but one may deal with it thanks to the possibility to average the field over macroscopic time interval which is mostly the case in for photovoltaic needs, where for example the light source is a distant star. 

\noindent Relations for substances under influence of those fields can be very complicated. There is a special case that can make life easier as well. If the material is \textit{isotropic} (all its properties are identical in every direction) they take a simple form of:

\noindent 

\noindent $\boldsymbol{j}=\sigma \boldsymbol{E}\boldsymbol{\to }$\textbf{ }Ohm's law

\noindent \textbf{}
\[\boldsymbol{D}\boldsymbol{=}\epsilon \boldsymbol{E}\] 
\textbf{}
\[\boldsymbol{B}\boldsymbol{=}\mu \boldsymbol{H}\] 
\textbf{}

\noindent Here $\sigma $ is called \textit{conductivity}, $\epsilon $ is a \textit{dielectric constant} and $\mu $ is \textit{magnetic permeability. }Normally, all of those are tensors. In the case of scalar conductivity we can separate macroscopic media in three different categories: conductors, semiconductors and isolators. The same goes for magnetic permeability. With $\mu <1$ the substance is said to be diamagnetic, $\mu >1$ paramagnetic and so with $\mu \gg 1$ ferromagnetic. Obviously this description is rather intuitive and treated as a general theory. For example for exceptionally strong fields the area of nonlinear optics is needed to be got into which provides higher power terms of fields in the above equations. Also, the case where we need to include relativistic effects by extracting previous values of E acting on charges is not included as well. The information will be expanded later when needed ,but if the reader wants to really expand following discussion understanding, it can be done via \eqref{GrindEQ__4_} \eqref{GrindEQ__5_}.557897406557897406MKMax Kliczkowski5578974061736006538Add illustrations and, equations numbers in text.

\noindent 


\subparagraph{ Boundary conditions at discontinuity}

\noindent 

\noindent For that Maxwell's equations are only stated in continuous regions we need to clarify the description with including the discontinuities in the medium. As so, the field vectors will be discontinuous, while $\rho $ and \textbf{j} will transform to different quantities available on the 557897407surface557897407MKMax Kliczkowski5578974071736006537Add boundary conditions with images!.

\noindent 


\subparagraph{ Energy of electromagnetic field, the Poynting vector }

\noindent 

\noindent In 

\noindent 


\subparagraph{ Wave equation}

\noindent 


\subparagraph{ Scalar waves and wave packets}

\noindent 


\subparagraph{ Vector waves}

\noindent 


\subparagraph{ Refraction and reflection of a classical electromagnetic wave }

\noindent 

\noindent 

\noindent     


\paragraph{ Quantum dots and nanostructures}

\noindent 
\section{}


\subparagraph{ Fabrication technology}

\noindent 


\subparagraph{ Carriers in Quantum Dots}

\noindent 

\noindent 


\subparagraph{ Colloidal Quantum Dots}

\noindent 

\noindent CQDs are semiconductor crystals of the nanometre-scale size, with the diameter of less than twice the Bohr radius size, which are restricted with surfactant molecules and deployed in a solution. They have manifested to provide a development of numerous types of optoelectronic devices including photodiodes and PV devices. The properties of CQDs are easily adjusted by changing the volumetric features of nanoparticles similarly to metallic nanoparticles in plasmonic transport. \eqref{GrindEQ__6_} \eqref{GrindEQ__7_}

\noindent 

\begin{enumerate}
\item  \textbf{Quantum dot size.}
\end{enumerate}

\noindent With changing the size of the nanoparticle we can control the band gap over significant range of spectrum. Yet, practically, the width of the nanoparticle is estimated from the energy band gap. \eqref{GrindEQ__8_}

\noindent 

\begin{enumerate}
\item  \textbf{Quantum dot shape}
\end{enumerate}

\noindent \textbf{}

\noindent The geometry in physics plays an important role, there is no question about that. It is obviously not different for Quantum Dots. The quantum confinement strongly depends on the shape and dimensional geometry. The ability to control the growth of certain form allows the nanoparticles to exhibit a variety of properties. 

\noindent As it is the introduction part we will proceed to investigate different forms later.

\noindent 


\subparagraph{ Chemical properties control.}

\noindent 

\begin{enumerate}
\item  \textbf{Ligand exchange}
\end{enumerate}

\noindent \textbf{}

\noindent While long-chain surfactants enable us to have a firmly stable CQD solution they are an insulating medium as well. \eqref{GrindEQ__9_} \eqref{GrindEQ__10_} To achieve massive improvements in film properties shall we partially or fully exchange the initial ligands in solution. The first studies of reactions to exchange the ligands were based on replacing the long-chain ones with organic short molecule structures (butylamine or pyridine \eqref{GrindEQ__11_} \eqref{GrindEQ__12_}) Also, the metal chalcogenide complex for these cases were tested widely and also quite successfully deployed. \eqref{GrindEQ__13_}

\noindent 

\begin{enumerate}
\item  \textbf{}

\item \textbf{ Alloying and Doping. }
\end{enumerate}

\noindent \textbf{}

\noindent The inner relationship between the band gap and fraction of nanocrystal particles in solution gives nonlinear characteristics and shall be studied to definitely provide the optimum. The distribution can determine precisely if the alloy is graded or homogeneous. \eqref{GrindEQ__14_}\textbf{}

\noindent Doping CQDs with other nanoparticles can also allow different strategic fabrication of PV structure than using single type of CQDs. An example of such behaviour can be PbS CQDs doped with silver, which has led to increasing the solar cell efficiency via introducing an extra charge carrier in the film \eqref{GrindEQ__15_}. 

\noindent 

\begin{enumerate}
\item  \textbf{Core/Shell Quantum Dots. }
\end{enumerate}

\noindent \textbf{}

\noindent Easy desorption of the ligands from the surface can be limited by more convenient surface passivation through the introduction of extra material shell- a core/shell structure. It can provide an isolation that would allow the core atoms to last significantly longer than without it.  \textbf{}

\noindent 

\noindent Again, the detailed part shall be provided in the next chapters. 

\noindent \textbf{}

\begin{enumerate}
\item \textbf{ Fully inorganic solar cells. }
\end{enumerate}

\noindent 

\noindent \includegraphics*[width=3.39in, height=2.33in, keepaspectratio=false]{image10}Between all of the examples shown above, a new idea of using all-inorganic CQD Photovoltaics has marked its' extraordinary intro. The approach to brand new CQDs films production based on the replacement of organic ligands with halide anions has become the new track for scientist. There are some advantages of using such a method of omitting redox reactions, with reducing the inner density of trap states within the bandgap among them. The so-called ligand passivation can be a strategy to ultimately enhance device performance. \eqref{GrindEQ__16_} Shall we ponder only the working principles, by many means the scheme isn't much different from other approaches, but it could remind us more about the original, first generation PV devices rather than organic ligand-based ones. 

\noindent 

\noindent Within this rather modern idea, the PbS and PbSe CQDs are commonly used. The widely explored architecture is shown in (Figure 3). A numerous researches and their record results has shown that architecture to be very promising. \eqref{GrindEQ__17_}\textit{  }The passivation of ligands has been achieved by different fabrication methods, resulting in various models and their performances, yet based on the same idea. \eqref{GrindEQ__17_}\textit{ }\eqref{GrindEQ__18_}\textit{ }\eqref{GrindEQ__19_} In case of this paper we shall also investigate whether we can improve the scheme of those structures and create even statelier device prototype. \textbf{}

\noindent 

\noindent 


\paragraph{ Introduction to quantum description of light}

\noindent 


\paragraph{ Theory of photovoltaic devices}


\paragraph{ Solar cell parameters}


\paragraph{ Third generation solar cells, QDSCs and review}

\noindent 


\subparagraph{ QDSC in principle. }

\noindent 

\noindent The great achievement of these types of solar cells is the decoupling the charge generation and its' transport to different materials- the hole and electron transport layers. The effect of such a procedure is the decrease in recombination process and reduction of production costs.  Without that technique, the architecture is adequate to the standard PV solar cell device. The generation of electron-hole pairs precedes the injection of electrons from Conduction band of light harvesting material to electron acceptor layer and holes from valence band to hole acceptor layer. We call this process a charge separation. Nevertheless, even though we do regenerate QDs in the light harvesting area after charge separation, we still have to struggle with recombination processes and consider them as the significant performance wasters. \eqref{GrindEQ__20_}

\noindent 


\subparagraph{ Materials and performance development.}

\noindent 

\begin{enumerate}
\item  \textbf{Electron transporting materials (ETMs)} \textbf{}
\end{enumerate}

\noindent \textbf{}

\noindent They are used to support charge separation as briefly described above. The function proceeds as a support for QD in the charge generation layer as it transports charged electrons to the conductive substrate (usually FTO) \eqref{GrindEQ__21_} \eqref{GrindEQ__22_} \eqref{GrindEQ__23_}. The properties shall be as follows:\textbf{}

\begin{enumerate}
\item \textbf{ }Matching CB edge -- which will determine exciton generation and partial transfer efficiency.\textbf{}

\item \textbf{ }High electron mobility.\textbf{}

\item \textbf{ }Fine surface to provide suitable loading of particles.\textbf{}

\item \textbf{ }Suitable technological properties, such as stability, low cost etc. \textbf{}
\end{enumerate}

\noindent The most widely studied materials nowadays are TiO${}_{2}$ and ZnO. Materials based on first compound are of such importance due to their nontoxicity, chemical stability and low cost \eqref{GrindEQ__24_} \eqref{GrindEQ__25_}. Among them (TiO${}_{2}$-NP) based mesoporous layers are studied as photoanodes \eqref{GrindEQ__24_}. The highest efficiency ever recorder were based on that material \eqref{GrindEQ__26_} \eqref{GrindEQ__27_} \eqref{GrindEQ__28_}. Unfortunately, the probability for charge recombination in that kind of films is rather unsatisfactory.  Therefore, one provides one- dimensional structures such as nanorods and nanowires to improve smoother electron transport channel and decrease loses provided with recombination \eqref{GrindEQ__29_} \eqref{GrindEQ__30_} \eqref{GrindEQ__31_}. In 2007 it was shown that TiO${}_{2}$ nanotubes win over nanoparticles of the same material in case of transport capacity \eqref{GrindEQ__32_} \eqref{GrindEQ__33_}.  Thus, the research of such structures was accelerated and CdSe sensitisation was proposed \eqref{GrindEQ__34_}. After that, many different sensitisations were provided, in which also PbS QDs had their small part \eqref{GrindEQ__35_}. Although promising, the 1D structured TiO${}_{2}$ based anodes were unsatisfactory in case of PCEs comparing to standard nanoparticles (maximum of around 6\% \eqref{GrindEQ__31_}). The hypothesis was put on insufficient loading amount on 1D structured TiO${}_{2}$, therefore the focus should be put on improvement of that area. A group of researchers has also used graphene frameworks incorporated into TiO${}_{2}$ photoanode achieving 4.2\% PCE for QDSSCs. \eqref{GrindEQ__36_} By using double-layered film with particles of bigger volume, there has been achieved a high PCE of 4.92\%. A lot of dopants and different solutions, hybrid photoanode films with metal and non-metal ions and carbonaceous layers has been proposed to improve performance of PV devices. Also with the usage of hollow structure techniques, by creating nanotube arrays, the possibility of achieving a PCE of 6\% was shown. \eqref{GrindEQ__37_} Improvement light absorbance was achieved by using microporous TiO${}_{2}$ photoanode for PbS quantum dot sensitised solar cells achieving up to 3.5\% PCE performance. \eqref{GrindEQ__38_}

\noindent 

\noindent In case of ZnO based photoanodes the differencing property is a higher electron mobility and better conduction band edge. With them, the achievement of higher V${}_{oc}$  is more probable \eqref{GrindEQ__39_} \eqref{GrindEQ__40_}. Unfortunately, the chemical stability of those films is significantly lower. Similarly to the former, the nanoparticles were widely studied. \eqref{GrindEQ__41_} \eqref{GrindEQ__42_} \eqref{GrindEQ__43_} \eqref{GrindEQ__44_} As an example, the CdS sensitised ZnO nanoparticles were used to construct a photoanode, which allowed the achievement of 4,46\% PCE. They used TiO${}_{2}$ passivation of ZnO surface to improve the PCE of same sensitized CdS/ZnO QDSCs by more than 2\%, achieving 4.68\%. \eqref{GrindEQ__43_} Similarly, the 1D structures for ZnO can be crystallised. The distinction from the former is the ease of its' development \eqref{GrindEQ__41_} \eqref{GrindEQ__43_}. The ZnO nanowires sensitised with CdSe constructed in 2007 were the launch of the technology development and allowed Young et al. achieve PCE of 4.15\% \eqref{GrindEQ__45_} \eqref{GrindEQ__46_} \eqref{GrindEQ__47_} \eqref{GrindEQ__48_}. Many more one dimensional structures of ZnO based photoanodes were used to research maximal performance but it was shown that the problem again concerns the surface area \eqref{GrindEQ__49_}. For example, it was shown that modification of surface of ZnO NRs with TiO${}_{2}$ improves PCE from 1.54\% up to 3.14\% \eqref{GrindEQ__50_}. The second strategy to improve the performance of such material based structures is double layer replacement with two different 1D structures such as NR on bottom and TP on top.  However, the PCEs of ZnO based films still stay behind TiO${}_{2\ }$ones. \eqref{GrindEQ__51_} \eqref{GrindEQ__52_} The Al/Cl hybrid doping allowed to use ZnO in IR spectrum. \eqref{GrindEQ__53_}

\noindent There are other kinds of ETMs researched as well.  Using big tandem structures has manifested superb PCE through simulations. \eqref{GrindEQ__54_} The doping of synergistic fullerene electro transport layer has proven to increase Solar Cell performance \eqref{GrindEQ__55_} A series of materials such as SnO${}_{2}$, NiO, BiVO${}_{4}$, Zn${}_{2}$SnO${}_{4}$,BaTiO${}_{3}$, CoO has been incorporated to QDSC manifesting future potential. \eqref{GrindEQ__37_} High performance has been achieved using CdS thin films as single-source precursors to ETM layer providing over 8\% PCE. \eqref{GrindEQ__56_}

\noindent \includegraphics*[width=6.23in, height=1.62in, keepaspectratio=false]{image11}All of the above materials are n-type semiconductors. Furthermore shall we proceed to introduce the p-type metal oxides and p-type Quantum dot sensitised cells. Typically, the NiO semiconductor is widely used. The promising and comprehensive material that can be implemented is CuSCN, which is an inorganic of p-type. Of course, the extraction of a hole from light harvesting layer to a redox couple will be rather slower than in case of electron. \eqref{GrindEQ__31_} \eqref{GrindEQ__57_}. Therefore, being the potential solution to this problem, the p-type semiconductor based layers are beginning to leave their mark in PV devices. As expected, the most accurate application of that type of films would be the tandem configuration construction, which contains both n-type and p-type QDSCs to overcome so called a Shockley-Queisser limit. However appealing, the efficiencies of p-type QDSCs are yet to be enhanced. \eqref{GrindEQ__58_}

\noindent 

\begin{enumerate}
\item  \textbf{Hole transporting material (HTM) layers.}
\end{enumerate}

\noindent 

\noindent The electrolyte or HTM is crucial to QDSCs as well. The properties of such film shall be:

\noindent 

\begin{enumerate}
\item  Low corrosivity.

\item  Redox potential appropriate to regenerate QDs and maintain fine V${}_{oc}$. 

\item  High conductivity through ions.

\item  Stability and transparency in visible spectrum.

\item  Possibility to fully regenerate. \eqref{GrindEQ__58_}
\end{enumerate}

\noindent The common electrolytes:

\noindent 

\begin{enumerate}
\item  Liquid electrolytes

\item  Quasi-solid state electrolytes

\item  All-solid state QDSCs
\end{enumerate}

\noindent were described comprehensively in the following research papers \eqref{GrindEQ__58_} \eqref{GrindEQ__59_} \eqref{GrindEQ__60_}

\noindent Although using Sb${}_{2}$Se${}_{3}$ as a thin light harvesting film, a group of researchers has also used PbS colloidal quantum dots as a hole transport layer achieving 6.5\% certified PCE in 2017. \eqref{GrindEQ__61_} In 2016 the usage of graphdiyne (a novel large $\piup$ -conjugated carbon hole transporting material) was used for an efficient hole transport layer for solar cells based on PbS-EDT colloidal quantum dots. \eqref{GrindEQ__62_} Before that, in 2015, colloidal CuInS${}_{2}$ QDs were layered to hole transporting solution and even though the scientist did their research on Perovskite solar cell, they accomplished to get almost 8.5\% PCE. \eqref{GrindEQ__63_} The extended device stability and a rise to 10.6\% of PCE was certified using CQD solar cells using P3HT as a hole transport material. \eqref{GrindEQ__57_}

\noindent \textbf{}

\begin{enumerate}
\item \textbf{ QD sensitizers }
\end{enumerate}

\noindent \textbf{}

\noindent The main part of our PV device is QDs. The ability of harvesting interfering light is a crucial component of such device. \eqref{GrindEQ__64_} \eqref{GrindEQ__65_}Therefore, the ideal nanostructures should exhibit certain properties:

\noindent 

\begin{enumerate}
\item  A higher CBE than the one in the electron transport material and lower than in hole transport material to provide effective charge separation. 

\item  Obviously, as in all semiconductor PV devices, we shall provide a material with accurate band gap, ideal for our purpose. The reason is clear, we are interested in superb absorption in wide range of solar spectrum.

\item  The stability is the crucial property as well. 

\item  From the technological point of view- the simple preparation and of course low toxicity would be rather expected. 
\end{enumerate}

\noindent Therefore, we will now examine methods to deposit them and the differences between certain QDs materials. 

\noindent Because of QDs being inorganic and possessing larger size than simple molecular dyes, they are much more difficult to tether onto metal oxide to form a high quality monolayer. Therefore high QD-loading amount is rather high to achieve. \eqref{GrindEQ__66_} We can difference the deposition \includegraphics*[width=3.98in, height=0.96in, keepaspectratio=false]{image12}methods by putting them into two categories: \textit{in situ }and \textit{ex situ}. In the first one, the QDs are grown directly on the surface of the metal oxide substrate, being created using an ionic precursor. We can include chemical bath deposition(CBD) and successive ionic layer adsorption and reaction(SILAR) into this category. The second approach bases on pre synthesising of QDs and then depositing them onto the metal oxide. Easily processable and finely reproductible, the in situ method has been used more widely than its' counter. In that kind of deposition we can control not only the QD-loading amount but also size distribution. Unfortunately, the achieved density of trap states is rather high, therefore the obtained maximal PCE is about 7\% \eqref{GrindEQ__67_} \eqref{GrindEQ__68_}. In this case, excluding the QDs growth process, we have to accurately deploy them onto the surface. The most common methods are: direct absorption, linker-molecule-assisted self-assembly, electrophoretic deposition. \eqref{GrindEQ__58_} \eqref{GrindEQ__69_}The ligand part in QDs is, as mentioned before, the important part of obtaining high efficiency of QDSCs. \eqref{GrindEQ__70_} The comparation of halide ligands in PbS CQDs for field effect transistors has been made by researchers in 2018. \eqref{GrindEQ__71_} The capping-ligand-induced self-assembly method was the clue for TiO${}_{2}$ photoanodes. \eqref{GrindEQ__58_} Nevertheless, researchers haven't discovered a satisfying approach yet. In 2012, the deposition method were optimised through using ligand-exchange techniques. \eqref{GrindEQ__72_} \eqref{GrindEQ__73_}Thanks to that, the PCE record of 13\% was achieved \eqref{GrindEQ__74_} Not only CLIS deposition has been used. The aqueous solution provided the simplification of QDs fabrication with shorter ligands. \eqref{GrindEQ__75_}Some relatively high PCEs have been achieved with that method. \eqref{GrindEQ__58_} Then, the organic molecules usage allowed scientist in 2015 to achieve 300\% times PCE compared to common creation of PbS quantum dots. \eqref{GrindEQ__76_} The certified PCE of 11.21\% has been achieved via so called ``solvent curving'', which is the simplified method of PbS QDs fabrication processing. \eqref{GrindEQ__77_} The passivation of PbS QD surface with L-glutathione was used to produce QDSCs with promising results. \eqref{GrindEQ__78_} 10.6\% PCE was achieved thanks to solvent-polarity-engineered halide passivation. \eqref{GrindEQ__79_}

\noindent Up to date, binary QDs have also been used as sensitizers. There are many examples but the most common are: InP, InAs, CdS, CdSe, CdTe, Ag${}_{2}$S and with them, the most interesting one considered in our case- PbS. \eqref{GrindEQ__80_} \eqref{GrindEQ__81_} \eqref{GrindEQ__82_} \eqref{GrindEQ__83_} \eqref{GrindEQ__84_} \eqref{GrindEQ__58_} The main problem with them is to control and balance the narrower band gap and relatively high conduction band edge. For example, for PbS nanocrystals the band gap is narrow, but the conduction band edge is rather low, which causes the problem and has to be dealt with. In case of binary QDs we have to balance between the light harvesting efficiency and efficiency of charge injection. Mixing binary structures (for example PbS with PbSe) has been also proven to enhance the interesting efficiency. \eqref{GrindEQ__85_} \eqref{GrindEQ__86_}Treatment of PbS QDs with metal salts provided some advantages in CQD PV devices resulting in the increase of short circuit current and fill-factor. \eqref{GrindEQ__87_} The crucial part of success in getting high efficiency would also be suitable engineering of solvent. \eqref{GrindEQ__88_}

\noindent The other method to use QDs as sensitizers in PV devices is to create a Core/Shell QDs. Their unique properties have been mentioned before. The alignment in these provides ability to tune the light-absorption range, recombination processes and charge separation. Usage of them in QDSCs is rather modern. The first noticed implementation was created by Lee et al. in 2009. Through SILAR method, he achieved PCE of 4.22\%. \eqref{GrindEQ__89_} Yet the difficulties in creating specific materials may occur, because of inability to prepare a stable precursor for deposition methods. Core shell QDs can be classified into three categories: type I, reverse type I, and type II structures. \eqref{GrindEQ__90_} \eqref{GrindEQ__91_} \eqref{GrindEQ__92_}A great deal of detailed information can be possessed from that review \eqref{GrindEQ__58_}.

\noindent The massive perspective in PV devices has also been established by using Alloyed QDs. These allow us to create the non-linear band gap \eqref{GrindEQ__93_} \eqref{GrindEQ__94_} \eqref{GrindEQ__95_}. Stability in such materials is again much higher than in their constituents. A lot of scientific research was done in that area. The summary is satisfactorily described in publication below. \eqref{GrindEQ__58_} The review finely describes recent achievements and potential usage of alloyed QDs in the near future. 

\noindent We can also include dopants to QDs. There were a handful of propositions in the former review as well. Yet, the promising idea has been introduced with the usage of metallic nanoparticles dopants by considering plasmonic physical phenomena. \eqref{GrindEQ__96_} \eqref{GrindEQ__97_} \eqref{GrindEQ__60_}

\noindent 

\begin{enumerate}
\item  \textbf{Counter electrode}
\end{enumerate}

\noindent \textbf{}

\noindent As we may already expect, they are also playing an important role in achieving high performance of PV device. The electrodes have to catalyse reduction reaction. These shall pose properties as follows:

\noindent 

\begin{enumerate}
\item  Good conductivity

\item  High catalytic activity

\item  Fine stability, either chemical and physical
\end{enumerate}

\noindent We can also put them into four categories:

\noindent 

\begin{enumerate}
\item  Noble metals

\item  Metal chalcogenides

\item  Carbon materials

\item  Composite CEs
\end{enumerate}

\noindent Recent progress has been comprehensively described in those publications \eqref{GrindEQ__58_} \eqref{GrindEQ__98_} \eqref{GrindEQ__59_}

\noindent 


\subparagraph{ Recombination control}

\noindent 

\noindent Non-radiative charge recombination can seriously disable any performance of PV device. Because of defects states in QDs, the recombination is more serious than in normal DSCs. The recombination processes maintain standard paths, but of course they have their own specificity. \eqref{GrindEQ__58_} Controlling of recombination processes can be either through material engineering and interface engineering. \eqref{GrindEQ__58_} More specific information shall be provided later. 

\noindent 


\subparagraph{ Stability}

\noindent 

\noindent Controlling of stability can be achieved through material design and of course temperature adjustment. More specific information is also provided with the \eqref{GrindEQ__58_} review. 

\noindent 


\subsection{ Cell architecture}

\noindent \textbf{}


\subsection{ Device depositions and parameters}


\subsection{ Results analysis}


\subsection{ Future possibilities}


\subsection{ Conclusion and outlook}


\subsection{ References}

1. Insights, Energy. Global Energy Perspective 2019: Reference Case. s.l.~: McKinsey, 2019.2. Nasa. https://climate.nasa.gov/evidence/. [Online] 2019. 3. International Energy Agency. Renewables Information. 2018.4. Max Born, Emil Wolf. Principles of optics - electromagnetic theory of propagation, interference and diffraction of light. s.l.~: Cambridge University Press, 1999.5. Jackson, John David. Classical Electrodynamics. 6. Colloidal Quantum Dot Solar Cells. Graham H. Carey, Ahmed L. Abdelhady, Zhijun Ning, Susanna M. Thon, Osman M. Bakr, Edward H. Sargent. s.l.~: Chemical Reviews, 2015.7. Colloidal PbS nanocrystals with size-tunable near-infrared emission: observation of post-synthesis self-narrowing of the particle size distribution. M.A. Hines, G.D. Scholes,. s.l.~: Adv. Mater, 2003, Vol. 15.8. Experimental Determination of the Extinction Coefficient of CdTe, CdSe, and CdS Nanocrystals. Yu, W. W., et al. s.l.~: Chem. Mater, 2003.9. Prospects of Colloidal Nanocrystals for Electronic and Optoelectronic Applications. Talapin, D. V., et al. s.l.~: Chemical Review , 2009, Vol. 110.10. Colloidal Nanocrystals with Molecular Metal Chalcogenide Surface Ligands. Kovalenko, M. V., Scheele, M. and Talapin, D. V. s.l.~: Science, 2009, Vol. 324.11. Ultrasensitive Solution-Cast Quantum Dot Photodetectors. Konstantatos, G., et al. s.l.~: Nature, 2006, Vol. 442.12. Hybrid Nanorod-Polymer Solar Cells. Huynh, W. U. Dittmer, J. J. Alivisatos. s.l.~: Science, 2002, Vol. 295.13. Temperature-Dependent Hall and Field-Effect Mobility in Strongly Coupled All-Inorganic Nanocrystal Arrays. Jang, J. Liu, W. Son, J. S. Talapin. s.l.~: Nano. Lett. , 2014, Vol. 14.14. Composition-Tunable Alloyed Semiconductor Nanocrystals. Regulacio, M. D. and Han, M.-Y. s.l.~: Acc. Chem. Res., 2010, Vol. 43.15. Systematic Optimization of Quantum Junction Colloidal Quantum Dot Solar Cells. Liu, H., et al. s.l.~: Applied Physics Letters, 2012, Vol. 101.16. All-Inorganic Colloidal Quantum Dot Photovoltaics Employing Solution-Phase Halide Passivation. Zhijun Ning, Yuan Ren , Sjoerd Hoogland , Oleksandr Voznyy , Larissa Levina , Philipp Stadler , Xinzheng Lan , David Zhitomirsky , and Edward H. Sargent. s.l.~: Advanced Materials, 2012.17. Improving the Performance of PbS Quantum Dot Solar Cells by Optimizing ZnO Window Layer. Cheng, Xiaokun Yang. Long Hu . Hui Deng. Keke Qiao . Chao Hu . Zhiyong Liu . Shengjie Yuan . Jahangeer Khan . Dengbing Li. Jiang Tang . Haisheng Song . Chun. s.l.~: Nano-Mircro Lett, 2017.18. Imbalanced charge carrier mobility and Schottky junction induced anomalous current-voltage characteristics of excitonic PbS colloidal quantum dot solar cells. Lilei H, Andreas Mandelis, Xinzheng Lan, Alexander Melnikov, Sjoerd Hoogland, Edward H. Sargent. 155-165, s.l.~: Solar Energy Materials and Solar Cells, 2016, Vol. 155.19. Highly Efficient Flexible Quantum Dot Solar Cells with Improved Electron Extraction Using MgZnO Nanocrystals. Xiaoliang Zhang, Pralay Kanti Santra, Lei Tian, Malin B. Johansson, Ha�kan Rensmo and Erik M. J. Johansson. s.l.~: ACS Nano, 2017.20. I. Mora-Sero, S. Gimenez, F. Fabregat-Santiago, R. Gomez, Q. Shen, T. Toyoda and J. Bisquert,. s.l.~: Acc. Chem. Res, 2009, Vol. 42.21. M. Ye, X. Gao, X. Hong, Q. Liu, C. He, X. Liu, C. Lin,. 1217-1231, s.l.~: Sustainable Energy Fuels, 2017, Vol. 1.22. J. Xu, Z. Chen, J. A. Zapien, C. S. Lee, W. Zhang. 5337-5367, s.l.~: Adv. Matter., 2014, Vol. 26.23. Cao, J. Tian and G. 634-653, s.l.~: Phys. Chem. Lett., 2015, Vol. 6.24. Y. Bai, I. Mora-Sero, F. De Angelis, J. Bisquert and P. Wang. s.l.~: Chem. Rev, 2014, Vol. 114.25. Cao, J. Tian and G. s.l.~: J. Phys. Chem. Lett, 2015, Vol. 6.26. J. Du, Z. Du, J. S. Hu, Z. Pan, Q. Shen, J. Sun, D. Long,H. Dong, L. Sun, X. Zhong, L. J. Wan. s.l.~: J. Am. Chem. Soc., 2016, Vol. 138.27. K. Zhao, Z. Pan, I. Mora-Sero, E. Canovas, H. Wang, Y. Song, X. Gong, J. Wang, M. Bonn, J. Bisquert and X. Zhong,. s.l.~: J. Am. Chem. Soc, 2015, Vol. 137.28. S. Jiao, J. Du, Z. Du, D. Long, W. Jiang, Z. Pan, Y. Li and X. Zhong. s.l.~: J. Phys. Chem. Lett, 2017, Vol. 8.29. Z. Peng, Y. Liu, Y. Zhao, K. Chen, Y. Cheng and W. Chen. s.l.~: Electrochim. Acta, 2014, Vol. 135.30. A. Fatehmulla, M. A. Manthrammel, W. A. Farooq, S. M. Ali and M. Atif. 358063, s.l.~: J. Nanomater., 2015.31. Y. F. Xu, W. Q. Wu, H.S. Rao, H.Y. Chen, D.B. Kuang and C.Y. Su. s.l.~: Nano Energy, 2015, Vol. 11.32. Kamat, D. R. Baker. s.l.~: Adv. Funct. Mater, 2009, Vol. 19.33. A. Kongkanand, K. Tvrdy, K. Takechi, M. Kuno, P. V. Kamat. s.l.~: J. Am. Chem. Soc, 2008, Vol. 130.34. Q. Shen, A. Yamada, S. Tamura and T. Toyoda. s.l.~: Appl. Phys. Lett., 2010, Vol. 97.35. Z. Zhang, C. Shi, J. Chen, G. Xiao and L. Li,. s.l.~: Appl. Surf. Sci, 2017, Vol. 410.36. Graphene Frameworks Promoted Electron Transport in Quantum Dot-Sensitized Solar Cells . Yanyan Zhu, Xin Meng, Huijuan Cui, Suping Jia, Jianhui Dong, Jianfeng Zheng,Jianghong Zhao, Zhijian Wang, Li Li, Li Zhang, and Zhenping Zhu. s.l.~: Appl. Mater. Interfaces , 2014.37. Performance Improvement Strategies for Quantum Dot Sensitized Solar Cells: A Review. Zhonglin Du, Mikhail Artemyev,Jin Wang, and Jianguo Tang. s.l.~: Journal of Materials Chemistry A, 2019.38. Improved light absorbance and quantum-dot loading by macroporous TiO2 . Muhammad Abdul Basit, Muhammad Awais Abbas, Eun Sun Jung,Jin Ho Bang, and Tae Joo Park1. s.l.~: Matt. Chem. and Phys., 2017.39. M. Ye, X. Gao, X. Hong, Q. Liu, C. He, X. Liu and C. Lin,. s.l.~: Sustainable Energy Fuels, 2017.40. J. Xu, Z. Chen, J. A. Zapien, C. S. Lee, W. Zhang. s.l.~: Adv. Matter, 2014, Vol. 26.41. C. Li, L. Yang, J. Xiao, Y. C. Wu, M. Sondergaard, Y. Luo, D. Li, Q. Meng, B. B. Iversen. s.l.~: Phys. Chem. Chem. Phys, 2013.42. H. M. Cheng, K. Y. Huang, K.-M. Lee, P. Yu, S. C. Lin,J. H. Huang, C. G. Wu, J. Tang. s.l.~: Phys. Chem. Chem. Phys., 2012.43. J. Tian, Q. Zhang, E. Uchaker, R. Gao, X. Qu, S. Zhang, G. Cao. s.l.~: Energy Environ. Sci.,, 2013.44. J. Tian, L. Lv, X. Wang, C. Fei, X. Liu, Z. Zhao, Y. Wang and G. Cao. s.l.~: J. Phys. Chem. C,, 2014.45. J. Xu, X. Yang, H. Wang, X. Chen, C. Luan, Z. Xu, Z. Lu, V. A. Roy, W. Zhang, C. S. Lee. s.l.~: Nano Lett., 2011.46. P. Sudhagar, T. Song, D. H. Lee, I. Mora-Sero, J. Bisquert, M. Laudenslager, W. M. Sigmund, W. I. Park, U. Paik and Y. S. Kang,. s.l.~: J. Phys. Chem. Lett, 2011.47. H. Chen, W. Li, Q. Hou, H. Liu, L. Zhu,. s.l.~: Electrochim, 2011.48. Z. Zhu, J. Qiu, K. Yan and S. Yang. s.l.~: ACS Appl. Mater. Interfaces, 2013.49. Enhanced Carrier Transport Distance in Colloidal PbS Quantum Dot-Based Solar Cells Using ZnO Nanowires. Haibin Wang, Victoria Gonzalez-Pedro, Takaya Kubo, Francisco Fabregat-Santiago,Juan Bisquert,Yoshitaka Sanehira,Jotaro Nakazaki, Hiroshi Segawa. s.l.~: The Journal of Physical Chemistry, 2015.50. M. Seol, H. Kim, Y. Tak and K. Yong. s.l.~: Chem. Commun, 2010.51. J. Tian, Q. Zhang, E. Uchaker, Z. Liang, R. Gao, X. Qu, S. Zhang, G. Cao,. s.l.~: J. Mater. Chem. A, , 2013.52. K. Yan, L. Zhang, J. Qiu, Y. Qiu, Z. Zhu, J. Wang, S. Yang. s.l.~: J. Am. Chem. Soc, 2013.53. Activated Electron-Transport Layers for Infrared Quantum Dot Optoelectronics. Jongmin Choi, Jea Woong Jo, F. Pelayo Garc\'{i}a de Arquer, Yong-Biao Zhao, Bin Sun, Junghwan Kim, Min-Jae Choi, Se-Woong Baek, Andrew H. Proppe, Ali Seifitokaldani, Dae-Hyun Nam, Peicheng Li, Olivier Ouellette, Younghoon Kim, Oleksandr Voznyy, Sjoerd Hoogla. s.l.~: Adv. Matt., 2018.54. Tandem solar cells from solution-processed CdTe and PbS quantum dots using a ZnTe/ZnO tunnel junction. Ryan W. Crisp, Gregory F. Pach, J. Matthew Kurley, Ryan M. France, Matthew O. Reese, Sanjini U. Nanayakkara, Bradley A. MacLeod, Dmitri V. Talapin, Matthew C. Beard, and Joseph M. Luther. s.l.~: Nano Letters, 2017.55. ynergistic Doping of Fullerene Electron Transport Layer and Colloidal Quantum Dot Solids Enhances Solar Cell Performance . Mingjian Yuan, Oleksandr Voznyy , David Zhitomirsky , Pongsakorn Kanjanaboos , and Edward H. Sargent. s.l.~: Adv. Mat., 2014.56. High Performance PbS Colloidal Quantum Dot Solar Cells by Employing Solution-Processed CdS Thin Films from a Single-Source Precursor as the Electron Transport Layer. Long Hu, Robert J. Patterson, Yicong Hu, Weijian Chen, Zhilong Zhang, Lin Yuan,Zihan Chen, Gavin J. Conibeer, Gang Wang, and Shujuan Huang. s.l.~: Adv. fun. mat., 2017.57. P3HT as a Hole Transport Layer for Colloidal. Darren C. J. Neo, Nanlin Zhang, Yujiro Tazawa1, Haibo Jiang, Gareth M. Hughes. s.l.~: ACS Applied Materials \& Interfaces, 2016.58. Quantum dot-sensitized solar cells. Zhenxiao Pan, Huashang Rao,Iva�n Mora-Sero�, Juan Bisquert and Xinhua Zhong. s.l.~: Chem. Soc. Rev, 2018, Vol. 47/7659.59. Recent advances of critical materials in quantum dot--sensitized solar cells. A review. Jialong Duan, Huihui Zhang, Qunwei Tang, Benlin He, Liangmin Yu. s.l.~: Royal Society of Chemistry, 2015.60. Colloidal quantum dot based solar cells: from materials to devices. Jeong, Jung Hoon Song and Sohee. s.l.~: Nano Convergence, 2017.61. 6.5\% Certified Sb2Se3 Solar Cells Using PbS Colloidal Quantum Dot Film as Hole Transporting Layer. Chao Chen, Liang Wang, Liang Gao, Dahyun Nam, Dengbing Li, Kanghua Li, Yang Zhao, Cong Ge, Hyeonsik Cheong, Huan Liu, Haisheng Song, Jiang Tang. s.l.~: ACS Energy Letters, 2017.62. Graphdiyne: An Efficient Hole Transporter for Stable High-Performance Colloidal Quantum Dot Solar Cells. Zhiwen Jin, Mingjian Yuan, Hui Li, Hui Yang, Qing Zhou, Huibiao Liu, Xinzheng Lan, Mengxia Liu, Jizheng Wang,Edward H. Sargent, and Yuliang Li. s.l.~: Materials Views, 2016.63. Colloidal CuInS2 Quantum Dots as Inorganic Hole-transporting . Mei Lv, Jun Zhu, Yang Huang, Yi Li, Zhipeng Shao, Yafeng Xu, and Songyuan Dai. s.l.~: ACS Applied Materials \& Interfaces, 2015.64. M.Kouhnavard, S. Ikeda, N.A. Ludin,N. B. AhmadKhairudin, B. V. Ghaffari, M. A. Mat-Teridi, M. A. Ibrahim, S. Sepeaiand K. Sopian. s.l.~: Renewable Sustainable Energy Rev, 2014.65. P. V. Kamat, K. Tvrdy, D. R. Baker and J. G. Radich. s.l.~: Chem. Rev., 2010.66. M. A. Becker, J. G. Radich, B. A. Bunker and P. V. Kamat. s.l.~: J. Phys. Chem. Lett, 2014.67. J. G. Radich, N. R. Peeples, P. K. Santra and P. V. Kamat. s.l.~: J. Phys. Chem. C,, 2014.68. J. Tian, L. Lv, C. Fei, Y. Wang, X. Liu, Cao,. s.l.~: J. Mater. Chem. A , 2014.69. N. Guijarro, T. Lana-Villarreal, I. Mora-Sero, J. Bisquert and R. Gomez. s.l.~: J. Phys. Chem. C, 2009.70. Colloidal Quantum Dot Ligand Engineering for High Performance Solar Cells. Ruili Wang, Yuequn Shang, Pongsakorn Kanjanaboos,Wenjia Zhou, Zhijun Ning, Edward H. Sargent. s.l.~: Royal Society of Chemistry, 2016.71. Comparing Halide Ligands in PbS Colloidal Quantum Dots for Field- Effect Transistors and Solar Cells. Dmytro Bederak, Daniel M. Balazs, Nataliia V. Sukharevska, Artem G. Shulga, Mustapha Abdu-Aguye, Dmitry N. Dirin, Maksym V. Kovalenko, Maria A. Loi. s.l.~: ACS Applied nano materials, 2018.72. Z. Pan, H. Zhang, K. Cheng, Y. Hou, J. Hua and X. Zhong,. s.l.~: ACS Nano, 2012.73. H. Zhang, K. Cheng, Y. M. Hou, Z. Fang, Z. X. Pan, W. J. Wu, J. L. Hua, X. H. Zhong,. s.l.~: Chem. Commun, 2012.74. W. Wang, W. Feng, J. Du, W. Xue, L. Zhang, L. Zhao, Y. Li and X. Zhong. s.l.~: Adv. Mater, 2018.75. S. Peng, F. Cheng, J. Liang, Z. Tao and J. Chen. s.l.~: J. Alloys Compd., 2009.76. `Darker-than-Black' PbS Quantum Dots: Enhancing Optical Absorption of Colloidal Semiconductor Nanocrystals via Short Conjugated Ligands. Carlo Giansante, Ivan Infante, Eduardo Fabiano, Roberto Grisorio, Gian Paolo Suranna, Giuseppe Gigli. s.l.~: Journal of the American Chemical Society, 2015.77. High-Efficiency PbS Quantum-Dot Solar Cells with Greatly Simplified Fabrication Processing via ``Solvent-Curing''. Kunyuan Lu, Yongjie Wang, Zeke Liu, Lu Han, Guozheng Shi, Honghua Fang, Jun Chen, Xingchen Ye, Si Chen, Fan Yang, Artem G. Shulga, Tian Wu, Mengfan Gu, Sijie Zhou, Jian Fan, Maria Antonietta Loi,and Wanli Ma. s.l.~: Adv. Mat. , 2018.78. Passivation of PbS Quantum Dot Surface with L-glutathione in Solid-State Quantum-Dot-Sensitized Solar Cells. Askhat N. Jumabekov, Niklas Cordes, Timothy D. Siegler, Pablo Docampo, Alesja Ivanova, Ksenia Fominykh, Dana D. Medina, Laurence M. Peterc, Thomas Beina,. s.l.~: Applied Materials \& Interfaces, 2016.79. 10.6\% Certified Colloidal Quantum Dot Solar Cells via Solvent- Polarity-Engineered Halide Passivation. Xinzheng Lan, Oleksandr Voznyy, F. Pelayo Garc\'{i}a de Arquer, Mengxia Liu, Jixian Xu, Andrew H. Proppe, Grant Walters, Fengjia Fan, Hairen Tan, Min Liu, Zhenyu Yang, Sjoerd Hoogland, and Edward H. Sargent. s.l.~: ACS Nano Letters, 2016.80. A. Zaban, O. I. Micic, B. A. Gregg, A. J. Nozik. s.l.~: Langmuir,, 1998.81. S. Yang, P. Zhao, X. Zhao, L. Qu and X. Lai. s.l.~: J. Mater. Chem A., 2015.82. D. R. Baker, P. V. Kamat. s.l.~: Adv. Funct. Mater, 2009.83. C. H. Chang, Y. L. Lee. s.l.~: Appl. Phys. Lett.,, 2007.84. W.-T. Sun, Y. Yu, H.-Y. Pan, X.-F. Gao, Q. Chen and L.-M. Peng. s.l.~: J. Am. Chem. Soc, 2008.85. Enhanced mobility in PbS quantum dot films via PbSe quantum dot mixing for optoelectronic applications. Long Hua, Shujuan Huangb, Robert Pattersonb, and Jonathan E. Halperta. s.l.~: Journal of Materials Chemistry C, 22019.86. Mixed-quantum-dot solar cells. Zhenyu Yang, James Z. Fan, Andrew H. Proppe , F. Pelayo Garc\'{i}a de Arquer, David Rossouw,Oleksandr Voznyy , Xinzheng Lan1, Min Liu1, Grant Walters1, Rafael Quintero-Bermudez, Bin Sun,Sjoerd Hoogland, Gianluigi A. Botton, Shana O. Kelley Ed. s.l.~: Nature Communications .87. Photovoltaic Performance of PbS Quantum Dots Treated with Metal Salts. Dong-Kyun Ko, Andrea Maurano, Su Kyung Suh, Donghun Kim, Gyu Weon Hwang, Jeffrey C. Grossman,Vladimir Bulovic�, Moungi G. Bawendi. s.l.~: ACS Nano, 2016.88. Solvent Engineering for High-Performance PbS Quantum Dots Solar Cells. Rongfang Wu, Yuehua Yang , Miaozi Li , Donghuan Qin , Yangdong Zhang and Lintao Hou. s.l.~: Nanomaterials, 2017.89. Lo, Y.-L. Lee and Y.-S. s.l.~: Adv. Funct. Mater.,, 2009.90. Parkinson, J. B. Sambur and B. A. s.l.~: J. Am. Chem. Soc.,, 2010.91. J. Yang, J. Wang, K. Zhao, T. Izuishi, Y. Li, Q. Shen, X. Zhong. s.l.~: J. Phys. Chem. C, 2015.92. Z. Pan, I. Mora-Sero, Q. Shen, H. Zhang, Y. Li, K. Zhao, J. Wang, X. Zhong and J. Bisquert. s.l.~: J. Am. Chem. Soc, 2014.93. M. D. Regulacio and M.-Y. Han. s.l.~: Acc. Chem. Res, 2010.94. Nie, R. E. Bailey and S. M. s.l.~: J. Am. Chem. Soc.,, 2003.95. X. H. Zhong, Y. Y. Feng, W. Knoll and M. Y. Han. s.l.~: J. Am. Chem. Soc, 2003.96. Efficiency Enhancement of PbS Quantum Dot/ZnO Nanowire Bulk-Heterojunction Solar Cells by Plasmonic Silver Nanocubes. Tokuhisa Kawawaki, Haibin Wang,Takaya Kubo, Koichiro Saito, Jotaro Nakazaki, Hiroshi Segawa,and Tetsu Tatsuma. s.l.~: ACS Nano, 2015.97. Broadband Enhancement of PbS Quantum Dot Solar Cells by the Synergistic Effect of Plasmonic Gold Nanobipyramids and Nanospheres. Si Chen, Yong jie Wang, Qipeng Liu, Guozheng Shi, Zeke Liu, Kunyuan Lu, Lu Han, Xufeng Ling, Han Zhang, Si Cheng, and Wanli Ma. s.l.~: Advanced Science News, 2017.98. Stable and efficient PbS colloidal quantum dot solar cells incorporating low-temperature processed carbon paste counter electrodes. Jincheng An, Xichuan Yang , Weihan Wanga, Jiajia Li , Haoxin Wang , Ze Yua, Chenghuan Gong , Xiuna Wanga, Licheng Sun. s.l.~: Solar Energy, 2017.99. Our World in Data . https://ourworldindata.org/energy-production-and-changing-energy-sources. [Online] 2019. 100. Meteotest AG. (1991-2010).

\noindent 
\section{}


\subsection{ Images}

\noindent 
\section{Figure 1. Energy sources contribution in world scale [97] 4Figure 2. Annual yearly sum of Global Horizontal Irradiation [110] 5Figure 3. The structure of all-inorganic solar cell based on PbS CQDs with optimized ZnO film and its I-V characteristics for two different ZnO film thickness. [15] 9Figure 4. Schematic illustration of energy levels in different core/shell QDs types. 13}

\noindent 
\section{}


\subsection{ Tables}

\noindent 
\section{}


\subsection{ Attachment}

\noindent \textbf{}

\noindent 

\noindent 


\end{document}

